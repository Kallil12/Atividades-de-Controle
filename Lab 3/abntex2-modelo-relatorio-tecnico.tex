%% abtex2-modelo-relatorio-tecnico.tex, v-1.9.7 laurocesar
%% Copyright 2012-2018 by abnTeX2 group at http://www.abntex.net.br/ 
%%
%% This work may be distributed and/or modified under the
%% conditions of the LaTeX Project Public License, either version 1.3
%% of this license or (at your option) any later version.
%% The latest version of this license is in
%%   http://www.latex-project.org/lppl.txt
%% and version 1.3 or later is part of all distributions of LaTeX
%% version 2005/12/01 or later.
%%
%% This work has the LPPL maintenance status `maintained'.
%% 
%% The Current Maintainer of this work is the abnTeX2 team, led
%% by Lauro César Araujo. Further information are available on 
%% http://www.abntex.net.br/
%%
%% This work consists of the files abntex2-modelo-relatorio-tecnico.tex,
%% abntex2-modelo-include-comandos and abntex2-modelo-references.bib
%%

% ------------------------------------------------------------------------
% ------------------------------------------------------------------------
% abnTeX2: Modelo de Relatório Técnico/Acadêmico em conformidade com 
% ABNT NBR 10719:2015 Informação e documentação - Relatório técnico e/ou
% científico - Apresentação
% ------------------------------------------------------------------------ 
% ------------------------------------------------------------------------

\documentclass[
	% -- opções da classe memoir --
	12pt,				% tamanho da fonte
	openany,			% capítulos começam em pág ímpar (insere página vazia caso preciso)
	oneside,			% para impressão em recto e verso. Oposto a oneside
	a4paper,			% tamanho do papel. 
	% -- opções da classe abntex2 --
	%chapter=TITLE,		% títulos de capítulos convertidos em letras maiúsculas
	%section=TITLE,		% títulos de seções convertidos em letras maiúsculas
	%subsection=TITLE,	% títulos de subseções convertidos em letras maiúsculas
	%subsubsection=TITLE,% títulos de subsubseções convertidos em letras maiúsculas
	% -- opções do pacote babel --
	english,			% idioma adicional para hifenização
	french,				% idioma adicional para hifenização
	spanish,			% idioma adicional para hifenização
	brazil,				% o último idioma é o principal do documento
	]{abntex2}

%\documentclass[12pt,a4paper,openany,oneside]{abntex2}
% ---
% PACOTES
% ---

% ---
% Pacotes fundamentais 
% ---
\usepackage{lmodern}			% Usa a fonte Latin Modern
\usepackage[T1]{fontenc}		% Selecao de codigos de fonte.
\usepackage[utf8]{inputenc}		% Codificacao do documento (conversão automática dos acentos)
\usepackage{indentfirst}		% Indenta o primeiro parágrafo de cada seção.
\usepackage{color}				% Controle das cores
\usepackage{graphicx}			% Inclusão de gráficos
\usepackage{microtype} 			% para melhorias de justificação
\usepackage{nameref}
\usepackage{chngcntr}
\counterwithin{equation}{chapter}
\counterwithin{figure}{chapter}
\counterwithin{table}{chapter}
% ---

% ---
% Pacotes adicionais, usados no anexo do modelo de folha de identificação
% ---
\usepackage{multicol}
\usepackage{multirow}
% ---
	
% ---
% Pacotes adicionais, usados apenas no âmbito do Modelo Canônico do abnteX2
% ---
\usepackage{lipsum}				% para geração de dummy text
% ---

% ---
% Pacotes de citações
% ---
\usepackage[brazilian,hyperpageref]{backref}	 % Paginas com as citações na bibl
\usepackage[alf]{abntex2cite}	% Citações padrão ABNT

% --- 
% CONFIGURAÇÕES DE PACOTES
% --- 

% ---
% Configurações do pacote backref
% Usado sem a opção hyperpageref de backref
\renewcommand{\backrefpagesname}{Citado na(s) página(s):~}
% Texto padrão antes do número das páginas
\renewcommand{\backref}{}
% Define os textos da citação
\renewcommand*{\backrefalt}[4]{
	\ifcase #1 %
		Nenhuma citação no texto.%
	\or
		Citado na página #2.%
	\else
		Citado #1 vezes nas páginas #2.%
	\fi}%
% ---

% ---
% Informações de dados para CAPA e FOLHA DE ROSTO
% ---
\titulo{Controle PID de Sistemas Dinâmicos: Sistema de Primeira Ordem}
\autor{José E. de A. Junior: 20170009356\\
\vspace{0.8cm}
Kallil de A. Bezerra: 20180154987\\
\vspace{0.8cm}
Rafael de M. M. Capuano: 20180010172\\
\vspace{0.8cm}
Victor K. C. Sousa: 20180155278\\}
\local{Brasil}
\data{9 de outubro de 2020}
\instituicao{%
  Universidade Federal do Rio Grande do Norte -- UFRN
  \par
  Departamento de Engenharia da Computação e Automação -- DCA}
\tipotrabalho{Relatório técnico}
% O preambulo deve conter o tipo do trabalho, o objetivo, 
% o nome da instituição e a área de concentração 
\preambulo{Modelo canônico de Relatório Técnico e/ou Científico em conformidade
com as normas ABNT apresentado à comunidade de usuários \LaTeX.}
% ---

% ---
% Configurações de aparência do PDF final

% alterando o aspecto da cor azul
\definecolor{blue}{RGB}{41,5,195}

% informações do PDF
\makeatletter
\hypersetup{
     	%pagebackref=true,
		pdftitle={\@title}, 
		pdfauthor={\@author},
    	pdfsubject={\imprimirpreambulo},
	    pdfcreator={LaTeX with abnTeX2},
		pdfkeywords={abnt}{latex}{abntex}{abntex2}{relatório técnico}, 
		colorlinks=true,       		% false: boxed links; true: colored links
    	linkcolor=blue,          	% color of internal links
    	citecolor=blue,        		% color of links to bibliography
    	filecolor=magenta,      		% color of file links
		urlcolor=blue,
		bookmarksdepth=4
}
\makeatother
% --- 

% --- 
% Espaçamentos entre linhas e parágrafos 
% --- 

% O tamanho do parágrafo é dado por:
\setlength{\parindent}{1.3cm}

% Controle do espaçamento entre um parágrafo e outro:
\setlength{\parskip}{0.2cm}  % tente também \onelineskip

% ---
% compila o indice
% ---
\makeindex
% ---

% ----
% Início do documento
% ----
\begin{document}

% Seleciona o idioma do documento (conforme pacotes do babel)
%\selectlanguage{english}
\selectlanguage{brazil}

% Retira espaço extra obsoleto entre as frases.
\frenchspacing 

% ----------------------------------------------------------
% ELEMENTOS PRÉ-TEXTUAIS
% ----------------------------------------------------------
% \pretextual

% ---
% Capa
% ---
\imprimircapa
% ---

% ---
% Folha de rosto
% (o * indica que haverá a ficha bibliográfica)
% ---
\imprimirfolhaderosto*
% ---

% ---
% Anverso da folha de rosto:
% ---

{
\ABNTEXchapterfont


% ---
% RESUMO
% ---

% resumo na língua vernácula (obrigatório)
\setlength{\absparsep}{18pt} % ajusta o espaçamento dos parágrafos do resumo
\begin{resumo}
O trabalho apresentado aqui mostra a teoria e simulações de três tipos, os controladores de primeira ordem, segunda ordem e, por último, um arranjo de dois controladores em cascata. Sempre usando o \textit{software} Matlab para essas simulações. 

De forma simplificada, vimos que é possível diminuir o sobressinal e o tempo para atingir a estabilidade com algumas configurações mais elaboradas de controle, combinando filtros ou até mesmo fazendo o controle em cascata, porém, para atingir resultados melhores é extremamente importante saber escolher os valores para as constantes, caso contrário o resultado final pode ser pior até que o de um sistema P simples.


 \noindent
 \textbf{Palavras-chaves}: Sistemas de controle. PID. Sistemas Dinâmicos.
\end{resumo}
% ---

% ---
% inserir lista de ilustrações
% ---
\pdfbookmark[0]{\listfigurename}{lof}
\listoffigures*
\cleardoublepage
% ---

% ---
% inserir lista de tabelas
% ---
\pdfbookmark[0]{\listtablename}{lot}
\listoftables*
\cleardoublepage
% ---

% ---
% inserir lista de abreviaturas e siglas
% ---
%\begin{siglas}
%  \item[ABNT] Associação Brasileira de Normas Técnicas
%  \item[abnTeX] ABsurdas Normas para TeX
%\end{siglas}
% ---

% ---
% inserir lista de símbolos
% ---
%\begin{simbolos}
%  \item[$ \Gamma $] Letra grega Gama
%  \item[$ \Lambda $] Lambda
%  \item[$ \zeta $] Letra grega minúscula zeta
%  \item[$ \in $] Pertence
%\end{simbolos}
% ---

% ---
% inserir o sumario
% ---
\pdfbookmark[0]{\contentsname}{toc}
\tableofcontents*
\cleardoublepage
% ---


% ----------------------------------------------------------
% ELEMENTOS TEXTUAIS
% ----------------------------------------------------------
\textual

% ----------------------------------------------------------
% Introdução (exemplo de capítulo sem numeração, mas presente no Sumário)
% ----------------------------------------------------------
\chapter*[Introdução]{Introdução}
\addcontentsline{toc}{chapter}{Introdução}

O controle automático tem desempenhado um papel vital no avanço da engenharia e da ciência. Além da sua importância em sistemas de veículos espaciais, sistemas robóticos, e semelhantes, o controle automático tornou-se uma importante parte da fabricação moderna e dos processos industriais. É possível citar o controle numérico de ferramentas e máquinas nas indústrias de manufatura, no projeto de sistemas de piloto automático em operações aeroespaciais e no projeto de carros e caminhões na indústria automobilística. O controle também é essencial no controle de pressão, temperatura, umidade e viscosidade nos processos industriais \cite{ogata}.

Os avanços na teoria e na prática do controle fornecem meios para atingir o desempenho desejado dos sistemas dinâmicos, melhorando a produtividade e aliviando o trabalho penoso de algumas atividades. Portanto, os engenheiros e cientistas precisam ter uma boa compreensão deste campo para poder aplicar e entender o que está sendo aplicado na prática.



% Ogata - cap 5 - Transient and Steady-State Response Analyses
% 234/978 - sistemas de primeira ordem

% ----------------------------------------------------------
% PARTE - preparação da pesquisa
% ----------------------------------------------------------
\chapter{Embasamento teórico}

O controlador PID esteve em uso por mais de um século em várias formas e aplicações. Já foi popular como um dispositivo puramente mecânico, também como um dispositivo pneumático e até eletrônico. Atualmente o PID é implementado em sistemas embarcados, e os microprocessadores são essenciais nessa tarefa \cite{timwescott1}

As três letras que compõem o PID vem de Proporcional, Integral e Derivativo, e cada um desses elementos tem uma tarefa diferente, portanto, causam diferentes efeitos na funcionalidade de um sistema. Num típico controle PID esses elementos são orientados por uma combinação de comandos do sistema e de respostas do sinal que está sendo controlado.


\section{Espaço de Estados}

O controle no Espaço de Estados é aplicável a sistemas de múltiplas entradas e múltiplas saídas, que podem ser lineares ou não-lineares, invariantes ou variantes no tempo e com condições iniciais nulas ou não. Considera-se que o estado de um sistema no instante $t_0$ é a quantidade de informação em $t_0$ que, em combinação com a entrada $u(t)$ em $t \geq t_0$, determina univocamente o comportamento do sistema para todo $t \geq t_0$.

Assim, temos a representação de um sistema dinâmico no espaço de estados com as seguintes equações:

\begin{equation}
	\dot{x(t)} = f(x(t), u(t),t)
	\label{eqn:equacao_estados}
\end{equation}

E

\begin{equation}
	y(t) = g(x(t), u(t),t)
	\label{eqn:equacao_saida}
\end{equation}

Em que a equação \ref{eqn:equacao_estados} é a Equação de Estados e a \ref{eqn:equacao_saida} é a Equação de Saída.

\begin{equation}
U(s) = K_pE(s)
\end{equation}

Dentre as características do controlador P podemos citar \cite{meneghetti1}:

\begin{itemize}
    \item O controlador proporcional é um amplificador com ganho ajustável (K);
    \item O aumento do ganho K diminui o erro do regime;
    \item Em geral, o aumento de K torna o sistema mais oscilatório, podendo gerar instabilidade;
    \item Melhora o regime e piora o transitório, sendo bastante limitado.
\end{itemize}


\section{Controlador Proporcional Integral (PI)}

O controle Integral é usado para adicionar uma precisão de longo termo ao \textit{loop} do controle, normalmente é usado com controle Proporcional. 

\begin{equation}
U(s) = \frac{(K_ps+K_i)}{S}E(s)
\end{equation}

Dentre as características do controlador PI podemos citar \cite{meneghetti1}:
\begin{itemize}
    \item Zera o erro de regime, pois aumenta o tipo do sistema em 1 unidade;
    \item É utilizando quando tempos resposta transitória aceitável e resposta em regime insatisfatória;
    \item Adiciona um polo em $p=0$ e um zero em $z = \frac{-K_i}{K_p} = \frac{-1}{\tau_i}$;
    \item Como aumenta a ordem do sistema, tempos a possibilidade de instabilidade do sistema original. Pode degradar o desempenho do controlador em malha fechada.
\end{itemize}

\section{Controlador Proporcional Derivativo (PD)}

É conhecido que o controle proporcional trabalha com o comportamento presente da planta, e o integral trabalha com o passado do sistema. O controlador derivativo tenta prever como será o comportamento do sistema, tentando adiantar a estabilidade. O componente diferencial é, resumidamente, a última posição conhecida menos o valor atual da posição.

\begin{equation}
U(s) = (K_p + K_ds)E(s)
\end{equation}

Dentre as características do controlador PD podemos citar \cite{meneghetti1}:

\begin{itemize}
    \item Leva em conta a taxa de variação do erro;
    \item É utilizado quando temos resposta em regime aceitável e resposta transitória insatisfatória;
    \item Adiciona um zero em $z = \frac{-K_p}{K_d}= \frac{-1}{\tau_d}$;
    \item Introduz um efeito de antecipação no sistema, fazendo com que o mesmo reaja não somente à magnitude do sinal de erro, como também à sua tendência para o instante futuro, iniciando, assim, uma ação corretiva mais cedo.
\end{itemize}

\section{Controlador Proporcional Integral Derivativo (PID)}

%\begin{figure}[h]
%	\centering
%	\includegraphics[scale=0.70]{planta.jpg}
%	\caption{Planta PID simples}
%\end{figure}

De forma resumida, o controlador PID, trabalhando numa malha fechada, tem a estrutura mostrada na imagem acima. A variável $e$ representa o erro, que é a diferença entre a saída desejada $r$ e a saída real $y$. Esse sinal de erro é alimentado para o controlador PID, que calcula tanto a derivada como a integral desse sinal de erro em relação ao tempo. O sinal do controlador, $u$, para a planta é igual ao ganho proporcional $K_p$ vezes a magnitude do erro mais a integral do ganho $K_i$ vezes a integral do erro mais a derivada do ganho $K_d$ vezes a derivada do erro.

\begin{equation}
u(t) = K_pe(t) + K_i\int e(t)\,dt + K_p\frac{de}{dt}
\end{equation}

Esse sinal de controle, $u$, é passado para a planta e a nova saída $y$ é obtida. Essa nova saída é enviada de volta e comparada à referência para que se encontre o novo sinal de erro. O controlador recebe esse novo sinal e atualizada a entrada do controlador. Esse processo continua enquanto o controlador estiver funcionando.

\section{Filtros}

Os filtros têm por objetivo tornar o controle implementado, diminuindo o sobressinal e o tempo para se alcançar o \textit{setpoint}.

\subsection{Filtro derivativo}

O filtro derivativo age no regime transitório. Esse tipo específico de filtro tenta prever alguns comportamentos, eliminando ruídos de alta frequência e suavizando o efeito do ruído no controle. No caso real a queda da água, por exemplo, poderia ser considerada como um ruído, porque essa queda causaria um distúrbio no tanque. Na simulação é um pouco mais difícil de notar esse tipo de comportamento, então, adiantando a parte prática, encontramos um pouco de dificuldade de ver o efeito dele no sistema.

\subsection{Filtro \textit{anti windup}}

Sistemas de controle podem sofrer com resposta transitória lenta e oscilatória se houver, como entrada, um sinal saturado, pois a malha de realimentação é desfeita quando esse valor não é considerado. Isso faz com que a saída demore muito para atingir o \textit{setpoint} e apresente oscilações frequentes, esses dois comportamentos são indesejáveis numa planta industrial. Então, para atacar esse problema existe o filtro \textit{anti windup}, que faz com que o controlador saia da região da saturação e volte para o comportamento esperado.

\section{Sistemas de segunda ordem}

Até aqui foram discutidos os conceitos necessários para a realização da primeira parte do experimento, para a segunda parte serão necessários alguns novos conceitos relacionados aos sistemas de segunda ordem. 

Os sistemas de segunda ordem são aqueles cujo modelo pode ser escrito por uma equação diferencial de segunda ordem, ou seja, os que possuem dois polos \cite{dan_madeira}.

O comportamento dos sistemas de segunda ordem é bastante sensível à mudança nos parâmetros, então é importante fazer uma classificação dos perfis existentes para esses sistemas, tentando entender melhor os fenômenos que ocorrem em cada caso. Durante as simulações foi notado que apesar de haver bastante matemática por trás dos modelos, muita coisa pode ser entendida de forma mais fácil de forma empírica, alterando os parâmetros e analisando a mudança nos gráficos.

Existem três tipos principais de sistemas de segunda ordem, são eles:
\begin{enumerate}
	\item Superamortecido: um polo na origem, proveniente da entrada degrau unitário e dois polos reais;
	
	\item Subamortecido: um polo na origem proveniente da entrada degrau unitário e dois polos complexos, vindos do sistema;
	
	\item Criticamente amortecido: um polo na origem proveniente da entrada degrau unitário e dois polos reais iguais.		
\end{enumerate}

\subsection{Sistemas de segurança}

No segundo experimento será necessário implementar um sistema de segurança que impeça a água de transbordar, mas a bomba também não pode trabalhar sem água, numa situação real ela poderia queimar. Então será feito um sistema de segurança que não permita nem que haja um transbordamento nem uma quebra do motor. 

Esse tipo de segurança é definido como sendo um conjunto de passos, na lógica de automação e controle, que devem existir para garantir a segurança de um equipamento, pessoa ou processo \cite{meneghetti1}. Nesses casos as travas lógicas, que são implementadas em \textit{software} e as travas físicas, que são \textit{hardware}, são usadas para garantir a segurança das pessoas e a continuidade da produção.

O sistema vai monitorar certas condições da planta, tentando mante-la dentro do limite operacional estabelecido, e quando condições de risco, ou que estiverem fora dos parâmetros criados, forem detectadas ele deve acionar alarmes, ou também pode atuar nos erros.

Na simulação as seguintes regras devem ser seguidas e implementadas no sistema de segurança:

\begin{itemize}
	\item Não deve ser enviado para o sistema uma tensão fora dos limites de $\pm 4V$;
	\item A bomba não pode trabalhar sem que haja líquido a ser bombeado;
	\item Nenhum dos dois tanques pode transbordar.
\end{itemize}

\section{Controle em cascata}

O controle em cascata é implementado quando a malha de controle simples não responde satisfatoriamente, principalmente em processos que possuem uma perturbação contínua em torno da variável que se deseja manipular, ou seja, o nível do tanque 2 com o constante recebimento de água a partir do tanque 1 pode se encaixar nessa situação. O controle em cascata consiste de um controlador primário que regula um controlador secundário, melhorando a velocidade de resposta e reduzindo os distúrbios causados pela malha secundária

No arranjo proposto pelo roteiro pode-se manipular a tensão enviada para a bomba e obter o nível desejado no tanque 1, que influencia na vazão de saída afetando a vazão de entrada do tanque 2. O objetivo é controlar o nível do segundo tanque a partir do primeiro, então com uma malha externa de controle é preciso captar o erro entre a referência desejada e o nível do tanque 2 e fornecer um valor para o tanque 1, alterando a tensão do motor e aumentando, ou diminuindo, o nível do segundo tanque.
\chapter{Desenvolvimento e resultados}



\chapter{Conclusão}

As simulações permitiram que fosse melhor entendido o comportamento dos controladores P, PI, PD e PID, e além desses, também implementamos os filtros, para entender o impacto que eles causam nos sinais dos controladores. Nos experimentos 2A e 2B o objetivo era controlar os níveis dos tanques 1 e 2, respectivamente, e no experimento 2C era pedido que apenas o segundo tanque fosse controlado. Para analisar os gráficos observamos o tempo de estabilidade, o tempo de subida, o sobressinal e a oscilação.

Foi observado que no sistema de primeira ordem, de acordo com nossas simulações, o tempo de estabilidade é menor em relação ao de segunda ordem, e também oscila menos. Além disso, esperávamos encontrar resultados melhores com o controle em cascata, porém o tempo de estabilidade aumentou e também ficou mais instável, isso pode ser um problema de \textit{tunning}, mas mesmo com pesquisa e vários testes empíricos não chegamos a um resultado realmente bom, que justifique o aumento de complexidade na implementação do controle em cascata numa planta industrial real, por exemplo.

Em todos os experimentos foi possível ver a ação proporcional, integral e derivativa, e as consequências de cada uma delas nos gráficos. Portanto, para diminuição de erro de regime, regula-se o $K_i$, para atuar na oscilação usamos o $K_p$ e para tornar o tempo de acomodação menor é importante usar o $K_d$.


\postextual

% ----------------------------------------------------------
% Referências bibliográficas
% ----------------------------------------------------------
\bibliography{abntex2-modelo-references}

\phantompart

\printindex

\end{document}
